%%% srs.tex --- 

%% Author: Baishampayan Ghose <b.ghose@gnu.org.in>
%% Version: $Id: srs.tex,v 0.0 2006/04/06 19:19:42 ghoseb Exp$


\documentclass[12pt,a4paper]{article}
%%\usepackage[debugshow,final]{graphics}
\newcommand{\VS}{\textit{Vyasa}}

%% SET ALL MARGINS 1 INCH ---------------------------------------
\oddsidemargin  0.0in
\evensidemargin 0.0in
\textwidth      6.27in
\headheight     14.5pt
\topmargin      0.0in
\textheight     8.90in
%% --------------------------------------------------------------

\usepackage{newcent}
\title {Vyasa -- A simple, multilingual text editor}
\author {Baishampayan Ghose, Aatif Haider, \textit{et al.} \\ D.Y. Patil College of Engineering \&{} Technology,
\\ Kolhapur, Maharashtra}

\begin{document}
%\maketitle
\tableofcontents

%%%%##########################################################################

\section{Introduction}
The {\bfseries Software Requirements Specifications} (SRS) of
\VS{} --- a simple, multilingual and cross-platform text editor
are laid out in this document. It is expected to be useful for
installation, configuration, documentation purposes by end-users and developers.

\subsection{Purpose}
This SRS documents the requirement specifications of the version 1.0 of
\VS. The requirements for installing and using \VS{} are explained in
detail here. Also the various parts of the code are explained here for
the future developer.

\subsection{Scope of the Project}
\VS{} is intended to be a very simple and easy to use text-editor. It
has excellent multilingual display capabilities which will enable the
end-user to use it for typing text in various languages. It supports
state-of-the-art Indic Language support via the Pango rendering
library. \VS{} is written in the Python programming language which is
also an excellent Object Oriented programming language and is incredibly
easy to use and extend. \VS{} uses the GTk+ GUI toolkit for the User
Interface which is also a Free \&{} Open Source cross-platform GUI
toolkit.
\VS{} can be a very good tool for people who want to use a light-weight
and feature rich text-editor for creating multilingual documents. It can
also be useful for students who want to learn Python programming.

\subsection{Definitions, Acronyms, Abbreviations \&{} References}
\begin{enumerate}
\item Python -- Python is an object-oriented, interpreted programming
  language with dynamic semantics [http://www.python.org/]
\item GTk+ -- GTk+ is a cross platform GUI toolkit [http://www.gtk.org/]
\item \VS{} -- \VS{} is a simple, multilingual text editor
  [http://vyasa.berlios.de/]
\item Pango -- Pango is a text rendering library for GTk+
\end{enumerate}

\subsection{Overview of the Document}
This document provides the users, developers of \VS{} a bird's eye view
of the software in general and information about installing, configuring
and enhancing \VS{} in particular.

The mechanics of installing \VS{} are provided in detail as well as the
requirements for using it.


\section{Overall Description}
\subsection{Product Perspective}
\subsection{Product Functions}
\subsection{User Characteristics}
\subsection{General Constraints}
\subsection{Assumptions \&{} Dependencies}

\section{Specific Requirements}
\subsection{External Interface Requirements}
\subsubsection{User Interfaces}
\subsubsection{Hardware Interfaces}
\subsubsection{Software Interfaces}
\subsubsection{Communication Interfaces}
\subsection{Functional Requirements}
\subsection{Performance Requirements}
\subsection{Design Constraints}
\subsection{Attributes}
\subsection{Other Requirements}




%%%%##########################################################################

\end{document}

%%% Local Variables: 
%%% mode: latex
%%% TeX-master: t
%%% End: 
